The Higgs boson was discovered in 2012 by the ATLAS and CMS collaborations. This heralded in a new era in high-energy physics. We know the Higgs boson exists, but have not precisely measured many of its properties. If these properties deviate from theory even slightly, there could be signs of new physics. Thus the goals of the experiment have shifted---from discovery of a predicted Standard Model particle to the precise measurement of its various characteristics. 

This thesis focuses on the measurement of one such Higgs boson property, its fiducial cross-section with the VBF production mode in the $WW^*\rightarrow\ell\nu\ell\nu$ channel. This production mode is rare, only about $7\%$ of Higgs production, and previous measurements of VBF Higgs cross sections in the $WW^*$ channel have associated uncertainties over $50\%$, far above those associated with ggF Higgs production~\cite{HWW2016}. The VBF Higgs production mode has a clear signal with its two associated jets, but even in a targeted fiducial region, contaminating backgrounds from $t\bar{t}$, ggF Higgs, and $Z\rightarrow\tau\tau$ are large. The $WW^*\rightarrow \ell\nu\ell\nu$ channel is chosen for its high branching ratio and clean lepton signature. We measure opposite flavor leptons (muons and electrons) to lower the contamination of Drell-Yan events. This analysis uses sophisticated statistical approaches to control and mitigate the effects from each background. We measure expected mis-identified lepton backgrounds with a data-driven method, and use several optimized BDT discriminants to isolate ggF, top, and diboson backgrounds. Our measurement also benefits from 139 fb$^{-1}$ of recorded luminosity, a factor four greater than that used in the previous analysis. Although our sample size has increased substantially, statistical uncertainties still outweigh the impact from any other systematic. Finally, we carefully choose a fiducial region based on signal region cuts that amplify VBF events above their expected backgrounds and use truth-level MC simulations to translate our cross-section measured in the reconstructed ATLAS detecter to one which can be compared to theoretical estimations and other experimental measurements. Table~\ref{tab:comparison} compares results from the 36.1 fb$^{-1}$ measurement in Ref.~\cite{HWW2016} to the 139 fb$^{-1}$ measurement detailed in this thesis. This measurement has much higher significance, with high enough certainty to be considered a discovery. 

\begin{table}[!h]
  \begin{center}
    \begin{tabular}{|l|c|c|}
       \hline
        Luminosity & 36.1 fb$^{-1}$    & 139 fb $^{-1}$ \\
      \hline
	Total uncertainty (\%) & +58/-56 & +23/-18 \\
	Statistical uncertainty (\%) & +48/-44 & +19/-16 \\
        Experimental uncertainty (\%) & $\pm$26 & $\pm$10 \\
        Theoretical uncertainty (\%) & $\pm$20 & +7/-6\\
        Observed $\sigma$ & 3.2 & 6.9\\
	\hline 
    \end{tabular}
    \caption{Comparing measured VBF $H\rightarrow WW^*$ cross sections using 36.1 fb$^{-1}$ and 139 fb$^{1}$ measured in this thesis \cite{HWW2016}.}
    \label{tab:comparison}
  \end{center}
\end{table}

This thesis details fiducial and inclusive cross-section measurements, which are the first part of a larger effort to measure differential cross section distributions. These differential measurements will further increase sensitivity to new physics by detecting potential effects in distributions that are not visible in total cross section values. The precision of fiducial and inclusive VBF cross-sections is certain to increase as the LHC records data in Run 3 and beyond. 
