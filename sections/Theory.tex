\section{Standard Model}
The Standard Model (SM) is the leading theory that describes interactions between particles at a subatomic scale. I begin with a brief summary of the SM itself beginning with brief descriptions of the fundamental particles and their forces before delving into a summarized mathematical formulation. Next, I discuss the history of the SM and crucial tests of the theory up until current work at the LHC.  I will then  outline some of the recent and current physics at the Large Hadron Collider (LHC) with a focus on Higgs boson measurements. Finally, I'll introduce my thesis' main focus, differential cross-section measurements of Vector-Boson-Fusion Higgs decaying into two W-bosons.

The Standard Model is one of the most successful scientific theories to date. Its predictions encompass all of the visible universe and continue to undergo careful testing. The SM combines three forces- electromagnetic, weak, and strong - into one elegant description. I'll follow in the steps of many before me and detail the theory through first introducing particles and forces. Next I will introduce the mathematical formalisms describing particle interactions.
\subsection{Particles and forces}
 The particles we define in high energy physics are the most minute portions of matter we're able to observe. They are generally considered point-like, have no internal structure, and cannot be further split. Each particle we can define has a unique set of quantum numbers and its own anti-particle (with the same mass and spin, but opposite electrical charge and quantum numbers).

Particles can split up into distinct groups- first bosons, with integer spin, and fermions, with half-integer spin. Bosons are 'force carriers' meaning when particles interact they exchange bosons. Fermions are at the heart of all conventional matter. Fermions can be split further into two categories- leptons and quarks. Quarks have fractional integer charge and interact strongly while leptons have integer charge and interact solely through the weak or electromagnetic forces. Both quarks and leptons are made of three generations of particles, each heavier and more unstable than the next. Charts showing quark/lepton families and their key quantum numbers are shown below. Each generation of quarks and leptons contains a particle doublet. Each lepton doublet contains a charged lepton and a neutrino while each quark doublet contains one $+2/3$ charged particle and one with a $-1/3$ charge. Each lepton and quark also has an anti-particle. All conventional, stable matter is made from the first generation of quarks and leptons.

There are four gauge bosons and one scalar boson predicted through the SM. These correspond to three fundamental forces in nature (the fourth, gravity, is so small on the scale of particle interactions as to not be considered). The strongest force on the subatomic scale is the strong force- this is mediated by the gluon- and works primarly to bind quarks together to form composite particles like protons or neutrons. The electromagnetic force is about 60x weaker than the strong force and is mediated by the photon. This force accounts for all electric interactions like that between an electron an an atomic nucleus. Finally, the weak force ($10^4$ weaker than the EM) facilitates $\beta$-decay and is mediated by massive Z and W bosons. Before going into more detail on the gauge bosons and the forces they mediate, I'd be remiss not to mention the Higgs boson. The only scalar boson predicted by the SM, it is has no charge or intrinsic spin. The Higgs gives mass to all other particles through Spontaneous Symmetry Breaking, which I'll expand on in later sections.
\begin{figure}[H]
	\centering
    \includegraphics[width=0.7\textwidth] {Pictures/SMparticles.png}\hspace{1cm}
    \caption{Three generations of quarks and leptons are shown along with all SM bosons \cite{PDG}}
    \label{fig:SMparticles}
\end{figure}
Photons are massless, spin-1 particles and mediate all electromagnetic interactions. They couple directly to any particle with electric charge- so quarks, leptons, and $W$/$Z$ bosons but not neutrinos. Since the photon is massless, the electromagnetic force can operate on infinitely long scales but it's force decreases with $1/r^2.$
 
Gluons are massless particles with no charge and a spin-1. They couple to color charges, which are a property of quarks. Each quark has one of three colors (RGB) while anti-quarks have "anti" versions of these. Colors ore conserved 'charges' just like electric charge. Quarks are never found alone as they couple so strongly to one another as to be confined in groups of two or three. These groups are "color-confined" meaning the quarks contain colors which add up to a color neutral sum. For instance, a two quark meson $u\bar{u}$ may have colors R and anti-R while a three quark hadron $uud$ (proton) may have colors R, G, and B. Gluons are different from photons in that they are not neutral to the charge they couple to. Gluons have two colors (8 total combinations) and can thus couple to each other. This makes the strong force distinct from the electromagnetic and has implications for long-distance interactions.
 
$W$ and $Z$ bosons, unlike gluons and photons, are massive. However, like their other gauge boson counterparts, they have spin-1 and mediate a charge (weak). $W^{\pm}$ mediates charged-current interactions which can violate flavor conservation between quarks and/or leptons and their neutrinos. $Z^{0}$ mediates neutral-current interactions which conserve flavor. $W^{\pm}$ bosons contain electric charge so can interact through EM as well. In addition, $W$ and $Z$ bosons contain weak charge (as do all fermions) so can self-couple as well as couple with all fermions. 
 
The Higgs boson will be further motivated and described in later sections but suffice to say it's a massive spin-0 particle which couples to all particles with mass (including itself). It doesn't mediate any force but is still an integral part of the SM.  

\subsection{Gauge Invariance}
According to Noether's theorem, for every continuous transformation of a field that leaves the Lagrangian invariant, there is a conserved current. Symmetries found in physical theories lead to conservation laws (and vice-versa). The Standard Model is a gauge theory built on symmetries such that all interactions between particles result from requiring the theory to be invariant under local gauge transformations. Each part of the Standard Model- from quantum electrodynamics (QED) to quantum chromodynamics (QCD) - is a gauge theory on its own, which simply means they have gauge invariance symmetries. In this section I'll step through the basic mathematic formalism for QED, QCD, and the combined electro-weak theory to illustrate the physical ramifications of gauge invariance and set the stage for the Higgs mechanism. The following sections are written with guidance from text \cite{HalzenMartin}. 

\subsubsection{Quantum Electrodynamics}
Quantum electrodynamics (QED) is the first, and simplest, physical gauge theory, describing how light and matter interact even under relativistic conditions. The theory produces extremely good agreement with experiment due to the success of perturbative solutions and entire textbooks are dedicated to its motivation and calculated predictions. Here I will generate the full QED Lagrangian by imposing local gauge invariance on the Lagrangian of a free fermion. 

First, the Dirac Lagrangian describes a free fermion of mass \textit{m}
\begin{equation}
\mathcal{L} = i \psi \gamma^\mu \partial_ \mu \psi - m\bar{\psi}\psi,
\end{equation}
where $\psi$ is a Dirac spinor and $\gamma^\mu$ represent the Dirac matrices. To demonstrate local gauge invariance we need to transform
\begin{equation}
\psi(x) \rightarrow e^{i\alpha(x)}\psi(x) 
\end{equation}
where $\alpha(x)$ depends on space and time arbitrarily. Directly substituting this into our Lagrangian shows that $\mathcal{L}$ is not invariant, and the $\partial_\mu$ term breaks this
\begin{equation}
\partial_\mu \psi \rightarrow e^{i\alpha(x)}\partial_\mu\psi + ie^{i\alpha(x)}\partial_\mu \alpha
\end{equation}
In order to mandate the theory is invariant we need to change this term to the "covariant derivative" $D_\mu$ which transforms 
\begin{equation}
D_\mu\psi \rightarrow e^{i\alpha(x)}D_\mu\psi . 
\end{equation}	 
In order to transform as such the "covariant derviative" has to contain a vector field $A_\mu$ and this field must transform so as to cancel with the unwanted part of the transformed $D_\mu$. 
\begin{equation}
D_\mu \equiv \partial_\mu - ieA_\mu
\end{equation}
where 
\begin{equation}
A_\mu \rightarrow A_\mu + \frac{1}{e}\partial_\mu \alpha
\end{equation}
Now the original Dirac equation is replaced with the following:
\begin{equation}
\mathcal{L} = \bar{\psi}(i\gamma^\mu\partial_\mu-m)\psi + e\bar{\psi}\gamma^\mu\psi A_\mu.
\end{equation}
By requiring local gauge invariance we've introduced a gauge field $A_\mu$ which couples to the Dirac particle just as the photon. In fact, if we take this as the photon gauge field and so add a kinetic energy term (which is also local gauge invariant!) we find the Lagrangian of QED. 
\begin{equation}
\mathcal{L} = \bar{\psi}(i\gamma^\mu\partial_\mu-m)\psi + e\bar{\psi}\gamma^\mu\psi A_\mu -\frac{1}{4} F_{\mu\nu}F^{\mu\nu}
\end{equation}
One can also see that adding a mass term to the Lagrangian for the new field ($\frac{1}{2}m^2A_\mu A^\mu$) would break gauge invariance, indicating the photon must be massless. From the free fermion Lagrangian, imposing local gauge invariance leads to the full interacting field theory of QED. This isn't a curiosity but an essential component of the theory, and the use of local gauge symmetry in deriving particle interactions doesn't end here.
\subsubsection{Quantum chromodynamics}
Quantum chromodynamics differs from QED in a few crucial ways. First, since quark color fields exist the QED $U$(1) gauge group is replaced with $SU$(3) and the free Lagrangian contains indices $j$ to denote the three color fields. 
\begin{equation}
\mathcal{L} = \bar{q}_j(i\gamma^\mu\partial_\mu - m)q_j.
\end{equation}
QCD also carries three quark flavors, which will be ignored here for simplicity. The QCD group is also non-Abelian since not all generators of the group commute with each other. These generators will be defined as $T_a$ where $=1,...,8$ and are linearly independent traceless $3\times3$ matrices (the Gell-Mann matrices $\lambda_a$ are conventional). The local color phase transformation required is thus 
\begin{equation}
q(x) \rightarrow e^{i\alpha_a(x)T_a}q(x)
\end{equation}
We can consider an infinitesimal phase transformation as 
\begin{equation}
q(x) \rightarrow [1+i\alpha_a(x)T_a]q(x), \\
\partial_uq \rightarrow (1+i\alpha_aT_a)\partial_\mu q + i T_a q \partial_\mu \alpha_a.
\end{equation}
Just as in the QED example, the last line breaks the invariance of $\mathcal{L}$ and we can proceed very similarly to the QED case by introducing a new gauge field (or in this case eight) called $G_\mu^a$ which transform 
\begin{equation}
G_\mu^a \rightarrow G_\mu^a - \frac{1}{g}\partial_\mu\alpha_a - f_{abc}\alpha_b G_\mu^c
\end{equation}
The last term added here is to cope with the non-Abelian nature of QCD (that not all the generators $T_a$ commute with each other). Just as in QED this invariance forms a covariant derviative 
\begin{equation}
D_\mu = \partial_mu + i g T_aG_\mu^a
\end{equation}
Replacing the derivative into our Lagrangian and adding a gauge invariant energy term for each of the $G_mu^a$ fields ($\frac{1}{4}G_{\mu\nu}^a G_{\mu\nu}^a$) yields the final gauge invariant QCD Lagrangian
\begin{equation}
\mathcal{L} = \bar{q}(i\gamma^\mu\partial_\mu - m)q - g(\bar{q}\gamma^\mu T_aq)G_\mu^a-\frac{1}{4}G_{\mu\nu}^aG_{\mu\nu}^a.
\end{equation}
Just as in the QED case, imposing local color phase invariance produced a new interacting field (or rather, eight) with a coupling specified as $g$. These are the gluon fields and just like photons, local gauge invariance requires them to be massless. Unlike the QED case, this Lagrangian's new kinetic term includes self-interaction between the gauge bosons - another key feature of QCD that is mandated by local color phase invariance. Gluons themselves must carry color charge and so self-couple - the structure of these self coupling terms and their single coupling strength $g$ are uniqely determined by gauge invariance. 

\subsubsection{Electroweak unification}
Thus far, I've summarized the theoretical backgrounds for symmetries (and so conserved quantities) in both quantum electrodynamics and chromodynamics. The weak force is the final Standard Model force and weak interactions are mediated by $Z$ and $W$ bosons. Unlike the gluons and photons of QCD and QED, these gauge bosons are massive. This is explained through spontaneous symmetry breaking of the electroweak force, which is descibed in the following section. Assuming that $W$/$Z$ bosons are massive, the weak force can be combined with QED and a central electroweak force (with its associated symmetries) can be described. 

The weak neutral current $J_\mu^{NC}$ as well as the charged currents $J_\mu$ and $J_\mu^\dagger$ can form a symmetry group of weak interactions. The charged currents correspond to the charged weak interaction with $W^\pm$ bosons while the neutral current is associated with the $Z^0$ boson.
\begin{equation}
J_\mu=\bar{\nu_L}\gamma_\mu\nu_L \\
J_\mu^\dagger =\bar{e_L}\gamma_\mu\nu_L
J_\mu^3=\frac{1}{2}\bar{\nu_L}\gamma_\mu\nu_L-\frac{1}{2}\bar{e_L}\gamma_\mu e_L
\end{equation}
L here denotes that these are left-handed spinors and particle names denote associated Dirac spinors. The charged currents can be written as a doublet using the Pauli spin matrices $\tau_i$ where $\tau_\pm=\frac{1}{2}(\tau_1\pm i\tau_2$ and
\begin{equation}
\chi_L=\begin{bmatrix}
        \nu  \\
        e^-
        \end{bmatrix}
\end{equation}
as 
\begin{equation}
J_\mu^+(x)=\bar{\chi_L}\gamma_\mu\tau_+\chi_L \\
J_\mu^+(x)=\bar{\chi_L}\gamma_\mu\tau_-\chi_L \\
J_\mu^3(x)=\bar{\chi_L}\gamma_\mu\frac{1}{2}\tau_i\chi_L \textnormal{with} i=1,23
\end{equation}
Now if a corresponding charge is defined $T^i=\int J_0^i(x)d^3x$ we have an $SU$(2)$_L$ algebra
\begin{equation}
[T^i,T^j]=i\epsilon_{ijk}T^k
\end{equation}
Unfortunately while a these currents create an $SU$(2) group, they don't correspond with the weak neutral current symmetry in a fairly obvious way, unlike the charged currents, the neutral current has a right handed component. One clear way to resolve this is to add in the electromagnetic current, as its a neutral current with left and right-handed components.
\begin{equation} 
j_\mu^{em}(x)=-\bar{e_R}\gamma_\mu e_R-\bar{e_L}\gamma_\mu e_L
\end{equation}
so the the electromagnetic current $j_\mu$ can be written using the coupling $e$
\begin{equation}
j_\mu=e j_\mu^{em} = e\bar{\psi}\gamma_\mu Q\psi
\end{equation}
with $Q$ the charge operator and generator of the $U$(1) symmetry group of EM. In order to "save" the symmetry of the weak neutral current, we can define an electromagnetic current $j_\mu^Y$, the weak hypercharge current, that is unchanged by $SU$(2)$_L$ transformations. We define a weak hypercharge $Y$ and its current $j_\mu^Y$
\begin{equation}
Q=T^3+\frac{Y}{2} \\
j_\mu^Y =\bar{\psi}\gamma_\mu Y \psi
\end{equation}
The combined current 
\begin{equation}
j_\mu^{em} = J_\mu^3+\frac{1}{2}j_\mu^Y
\end{equation}
now generates the symmetry group $U$(1)$_\gamma$ and so the electromagnetic interaction and weak interaction are combined into one $SU$(2)$_L$ $\times$ $U$(1)$_\gamma$. While unified into one enlarged group, the two forces still have independent coupling strengths. This brief introduction into electroweak unification is not the complete picture- EM and weak interactions still have to be unified. This is simple in the Standard model framework- electroweak currents just have to be coupled to vector bosons. In the electroweak $SU$(2)$_L$ $\times$ $U$(1)$_\gamma$ group there is an isotriplet of vector fields $W_\mu^i$ coupled with strength $g$ to the weak isospin current $J_\mu^i$ while a single vector field $B_\mu$ is coupled to the weak hypercharge current $j_\mu^Y$ with strength $g'/2$. The electroweak interaction can be defined
\begin{equation}
-i g (J^i)^\mu W_\mu^i-i\frac{g'}{2}(j^Y)^\mu B_\mu
\end{equation}
This summary of the unified electroweak force will be the starting point for a derivation of the Higgs boson and an explanation for mass of the weak force's vector bosons (and all fermion masses). The electroweak theory is unique in its calculability, even at higher order scales. Because of this, many deviations from theory could be observed at current energy scales- the theorectical uncertainties are low. The measurement central to my thesis probes for such discrepancies to electroweak theory. The mechanisms for this will be explained in the last section in this section. 

\subsubsection{Spontaneous Symmetry Breaking}
Unlike QED and QCD, the weak force is mediated by massive gauge bosons. Because of this, we can't apply the same gauge invariance prescription that we did in the last sections. If a mass term is added to the Lagrangian we break the gauge invariance we aimed to find. If we instead ignore the gauge invariance and add a mass term to the Lagrangian, all predictive power of the theory is lost due to unrenormalizable divergences. With "spontaneous symmetry breaking" we can gain massive gauge bosons while maintaining the integrity of the theory. In this section I first describe the "spontaneous symmetry breaking" mechanism is terms of an Abelian theory composed of complex scalar fields to illustrate the overall strategy. This mechanism is then applied to the non-Abelian electroweak theory to gain massive weak gauge bosons $W^{+/-}$ and $Z$, with the Higgs field appearing as a 'spontaneous' result.

The Lagrangian for a $U$(1) gauge symmetry 
\begin{equation}
\phi \rightarrow e^{i\alpha(x)}\phi
\end{equation}
As in the QED case, we introduce a gauge field $A_\mu$ and covariant derivative $D_\mu = \partial_\mu - ieA_\mu$ to obtain the gauge invariant Lagrangian
\begin{equation}
\mathcal{L} = (\partial^\mu+ieA^\mu)\phi^*(\partial_mu-ieA_\mu)\phi-\mu^2\phi^*\phi-\lambda(\phi^*\phi)^2-\frac{1}{4}F_{\mu\nu}F^{\mu\nu}.
\end{equation}
In this example if $\mu^2>0$ we gain back the QED Lagrangian for a charged scalar particle of mass $\mu$ - with an additional self-interaction term. However, if we take $\mu^2<0$ the potential $V(\phi^*\phi)=\mu^2\phi^*\phi-\lambda(\phi^*\phi)^2$ now has a non-zero vacuum expectation value (v.e.v.) and there's a set of equivalent minima shown in Figure \ref{fig:HiggsPotential}. Choosing one of these minima spontaneously breaks the potential's rotational symmetry. Next, we can perturbatively expand the field about a minima through
\begin{figure}[H]
    \centering
    \includegraphics[width=0.5\textwidth] {Pictures/HiggsPotential.png}\hspace{1cm}
    \caption{Higgs potential when $\mu^2<0$, choosing a minima spontaneously breaks the $U$(1) rotational symmetry \cite{HiggsPotential}}.
    \label{fig:HiggsPotential}
\end{figure}

\begin{equation}
\phi(x) =\sqrt{\frac{1}{2}[\nu+\eta(x)+i\xi(x)]}
\end{equation}
Substituting this perturbation gives the new Lagrangian
\begin{equation}
\mathcal{L}'=\frac{1}{2}(\partial_\mu\xi)^2+\frac{1}{2}(\partial_\mu\eta)^2-\nu^2\lambda\eta^2+\frac{1}{2}e^2\nu^A_\mu A^\mu-e\nu A_\mu \partial^\mu\xi-\frac{1}{4}F_{mu\nu}F^{\mu\nu} + \textnormal{interaction terms}.
\end{equation}
Three particles seem to emerge here: massless Goldstone boson $\xi$, massive vector $A_\mu$ with $m_A=e\nu$ and a massive scalar $\eta$ with $m_\eta=sqrt{2\lambda\nu^2}$. However, the number of particles does not correspond to the polarization degrees of freedom expected, a longitudonal polarization was added, creating an unphysical field. To eliminate the unphysical field we can substitute new set of fields 
\begin{equation}
\phi \rightarrow \sqrt{\frac{1}{2}(\nu+h(x))e^{i\theta(x)/\nu}}
\end{equation}
and 
\begin{equation}
A_\mu \rightarrow A_\mu + \frac{1}{e\nu} \partial_\mu\theta.
\end{equation}
Introducing these substitutions, the Goldstone boson field disappears
\begin{equation}
\mathcal{L} = \frac{1}{2}(\partial_\mu h)^2 -\lambda\nu^2h^2+\frac{1}{2}e^2\nu^2A_\mu^2-\lambda\nu h^3-\frac{1}{4}\lambda h^4+\frac{1}{2}e^2A_\mu^2h^2+\nu e^2A_\mu^2h-\frac{1}{4}F_{\mu\nu}F^{\mu\nu}
\end{equation}
Here the degrees of freedom before our substitutions remains the same and a massive boson $A_\mu$ is preserved along with a massive scalar $h$. The "Higgs mechanism" applied to a scalar field succeeded in creating a massive boson and determined the existence of a massive scalar boson. This same mechanism can be applied in the more complicated Standard Model electroweak field. Through electroweak symmetry breaking we not only gain massive gauge bosons and a massive scalar boson, (the Higgs!), but a way to calculate testable Standard Model predictions for many quantities. We start with the $SU$(2)$\times$$U$(1) gauge symmetry of electroweak interactions derived in the previous section. In order to gain masses for three gauge bosons and keep the photon massles we need at least 3 degrees of freedom added and a simple choice is the $SU$(2) doublet of scalar fields $\phi$, with four fields in an isospin doublet of weak hypercharge $Y=1$ 
\begin{equation}
\mathcal{L} =(D^\mu \phi)^\dagger(D_mu\phi)-\mu^2\phi^\dagger\phi-\lambda(\phi^\dagger\phi)^2
\end{equation}
where $\phi$ is a $SU$(2) doublet of complex scalar fields
\begin{equation}
\phi = \sqrt{\frac{1}{2}}\begin{bmatrix}
	\phi_1+i\phi_2  \\
	\phi_3+i\phi_4
	\end{bmatrix}
\end{equation}
Local gauge invariance can be achieved just as shown previously with the covariant derivative 
\begin{equation}
D_\mu = \partial_\mu + ig \frac{\tau_a}{2}W_\mu^a
\end{equation}
with three gauge fields designated with $W_\mu^a(x)$ and $a=1,2,3$. An infinitesimal transformation is defined
\begin{equation}
\phi(x)\rightarrow \phi'(x) = (1+i\alpha(x)\cdot\tau/2)\phi(x)
\end{equation}
so that we find a Lagrangian potential 
\begin{equation}
V(\phi) = \mu^2\phi^\dagger\phi+\lambda(\phi^\dagger\phi)^2.
\end{equation}
Once again if we choose the conditions $\mu^2<0$ and $\lambda>0$ there is rotational symmetry in our choice of vacuum expectation value. In this case choice of vev is limited. For the photon to remain massless, the vacuum must be invariant under $U$(1) (or electromagnetic) transformations, and not be charged in either direction (charge conservation). Thus the chosen minima to spontaneous break electroweak symmetry is
\begin{equation}
\phi_0  = \sqrt{\frac{1}{2}}\begin{bmatrix}
        0  \\
        \nu
        \end{bmatrix}
\end{equation}
Next, as before substituting the vacuum expectation value $\phi_0$ for $\phi(x)$ and expanding perturbativelyyields 
\begin{equation}
\phi(x) \rightarrow \begin{bmatrix}
        0  \\
        \sqrt{\frac{1}{2}(\nu+H(x))}
        \end{bmatrix}
\end{equation}
Fully expanding this term in the Lagrangian gives a complex and illuminating result, the Goldstone bosons have been consumed and there is only a Higgs field ($H(x)$) remaining. Next, masses for the vector bosons are found from expanding one key parameter in the Lagrangian
\begin{equation}
|(-ig\frac{\tau}{2}\cdot W_\mu-i\frac{g'}{2}B_\mu)\phi|^2=\sqrt{\frac{1}{2}}\begin{bmatrix}
        gW_\mu^3+g'B_\mu && g(W_\mu^1-iW_\mu^2)  \\
        g(W_\mu^1+iW_\mu^2) && igW_\mu^3+g'B_\mu
        \end{bmatrix} = \begin{bmatrix}
        0 \\
        \nu
        \end{bmatrix}|^2
\end{equation}
Expanding further and substituting $W^\pm =(W^1\pm iW^2)/\sqrt(2)$ gives the result
\begin{equation}
|(-ig\frac{\tau}{2}\cdot W_\mu-i\frac{g'}{2}B_\mu)\phi|^2= (\frac{1}{2}\nu g)^2W_\mu^+W^{-\mu}+\frac{1}{8}(W_\mu^3,B_\mu) \begin{bmatrix}
        g^2 && -gg' \\
        -gg' && g'^2
        \end{bmatrix} \begin{bmatrix}
        W^{\mu 3} \\
        B^\mu
        \end{bmatrix}
\end{equation}
It's immediately clear that there's a mass-term for the $W^\pm$, $M_W=\frac{1}{2}\nu g$. Masses for the photon and $Z$-boson are also apparent after the expansion of the last equations final term
\begin{equation}
\frac{1}{8}\nu^2(g^2(W^\mu_3)^2-2gg'W_\mu^3 B^\mu+g'^2B_\mu^2)=\frac{1}{8}\nu^2(gW_\mu^3-g'B_\mu)^2
\end{equation}
and using the substitutions 
\begin{equation}
A_\mu=\frac{g'W_\mu^3+gB_\mu}{\sqrt(g^2+g'^2)} \textnormal{ with  } M_A=0, \\
Z_\mu =\frac{gW_\mu^3-g'B_\mu}{\sqrt(g^2+g'^2)} \textnormal{ with  } M_Z=\frac{1}{2}\nu\sqrt(g^2+g'^2) .
\end{equation}

Now the Higgs field exists just in the previous example and the theory contains a massive scalar boson and three massive vector gauge fields - for each of the $W$ and $Z$ bosons. The Goldstone bosons have been consumed, their degrees of freedom used to give mass to the vector bosons. Choosing a ground state and so breaking the gauge symmetry doesn't eliminate this symmetry altogether, since the theory is still renormalizable. Fermion masses can also be derived from their interactions with the Higgs boson using this Lagrangian. These derivations can be used to predict masses of bosons and fermions and couplings to the Higgs boson. It's important to note that though the Higgs mechanism gives mass to all fermions and massive gauge bosons, it doesn't determine what the Higgs mass ought to be. This is left as an empirical input to the theory with which to calculate other observables. 

The Standard Model has been proven over decades to be an incredibly robust theory and Large Hadron Collider (LHC) is it's key testing ground. 

\section{LHC Physics/Phenomenology}
The Large Hadron Collider (LHC) is the foremost Standard Model testing ground and the the proton-proton collisions recorded through the ATLAS detector have created greater understanding of the fundamental constants discussed in the previous section. Fermion and gauge boson masses and couplings, including the mass of the Higgs boson, have been measured incredibly precisely. In the next chapter the mechanics of the LHC and ATLAS detector will be discussed, but first here I will introduce the motivations and observations of LHC physics. This section will begin with the mechanics of proton-proton collisions and their decay products, then introduce the concept of decay cross-sections and finally focus more closely on the Higgs boson and its properties.

The LHC was designed with one central goal- discover the missing Standard Model Higgs boson. The protons in the LHC collide at a center-of-mass energy of 14 TeV, but began at half that in 2010. The electroweak symmetry breaking scale was theoretically known to be between 100-1000 GeV and so probing at 7 TeV provided near certainty of finding either the Higgs or an inconsistency in the Standard Model. The motivation for a proton collider was multifaceted. Foremost, using the tunnels built for the electron-positron detector LEP with protons allows the collider to reach higher energies, as protons don't lose energy to synchroton radiation at the same scale as electrons. However, proton collisions have added complexity from their component quarks. Each parton carries some fraction of the momentum of the proton described by parton distribution functions. 

Figure \ref{fig:protonproton} shows a proton-proton collision schematic. In this example, from \cite{Butterworth}, the hard process comes from the up quark in each proton. "Hardness" refers to the fraction of momentum of the protons that is involved in the collision. In contrast, "soft" collisions are those from remaining partons in each proton and usually involve low momentum transfer. These soft collisions are considered the underlying event shown in the figure. Proton scatter of partons is the most common hard process at the LHC by far due to the high density of gluons in the proton and the scale of QCD couplings above electroweak. 
\begin{figure}[H]
        \centering
    \includegraphics[width=0.6\textwidth] {Pictures/protonproton.png}\hspace{1cm}
    \caption{Proton-proton collision example showing quark-gluon scattering and final state jet and Z-boson \cite{Butterworth}}.
    \label{fig:protonproton}
\end{figure}
Quarks and gluons emitted from the high energy hard scatter do not appear in the detector directly. QCD, in one of its key differences to QED, becomes stronger with larger distances. As it reaches high enough energy low energy gluons are radiated until partons are able to bind into color-neutral hadrons. These hadrons are seen collimated in groups in the detector as "jets". The energy and momentum of jets are used as reflections of the intial scattered partons. Various "jet algorithms" are used to determine the initial parton properties as reproducibility and precisely as possible. The jet algorithm used in this analysis will be described in detail in Chapter 3. However, it's important to note that the algorithm used by all LHC experiments, anti-$k_t$- is collinear and infrared safe, or no addected by small angle and soft scatterings that occur in a parton shower. Without these qualities, perturbation theory applied to the parton shower would find infinities at higher orders.

The cross-section (denoted $\sigma$) measures the probability that a certain process will occur in the collision of two particles, in our case protons. In high energy physics cross-sections are measured in inverse femtobarns. A barn is the cross-sectional area of a uranium nucleus and was named to describe the large target area needed to have direct strikes on a nucleus. Hence the expression "couldn't hit the broad side of a barn". Inverse femtobarns are used to measure the number of particle collision events per femtobarn area of a target and measures time-integrated luminosity. 

Hard scattering cross-sections in hadron-hadron collisions can be calculated using the QCD factorization theorem, and to leading-order these calculations are relatively simple. In the factorization theorem, developed by Drell and Yan, deep inelastic scattering parton model processes could apply to hadron-hadron collisions. The Drell-Yan process is the production of a massive lepton pair by quark-antiquark annihilation. According to the factorization theorem, a hadronic cross-section $\sigma(AB\rightarrow \mu^+\mu^-+X)$ could be calculated by weighting the Drell-Yan sub-process cross-section $\hat{\sigma}$ for $\bar{q}q\rightarrow\mu^+\mu^-$ with parton distribution functions $f_{q/A}(x)$ which come from deep inelastic scattering \cite{Campbell}:
\begin{equation}
\sigma_{AB} = \int dx_a dx_b f_{a/A}(x_a)f_{b/B}(x_b)\hat{\sigma_{ab\rightarrow X}}
\end{equation}
where $X$ represents the two resulting leptons and $ab$ the two annihilated quarks. This parton model provides good agreement with measured cross sections and so allows understanding of particlar hard scattering processes. Predictions for some key Standard Model processes are shown in Figure \ref{fig:crosssection}. Noting the logarithmic scales it's clear that the Higgs boson of mass 125GeV is orders of magnitude more numerous at the LHC than the Tevatron and that certain high mass particles like the $b$ quark and $W/Z$ bosons are produced at the LHC at high levels \cite{Campbell}. In addition, the plot shows cross-sections of particular Higgs decay modes. These will be discussed next.
 
\begin{figure}[H]
        \centering
    \includegraphics[width=0.6\textwidth] {Pictures/crosssections.jpg}\hspace{1cm}
    \caption{Predicted Standard Model cross-sections for the Tevatron and LHC \cite{Stirling}.}
    \label{fig:crosssection}
\end{figure}

Higgs production at the LHC occurs via four main processes: gluon-gluon fusion, vector-boson fusion, associated production with $W/Z$ bosons, and associated production with top or bottom quarks. The Feynmann diagrams for these processes are shown in Figure \ref{fig:FeynmannHiggs}.  

\begin{figure}[H]
        \centering
    \includegraphics[width=0.7\textwidth] {Pictures/FeynmannHiggs.png}\hspace{1cm}
    \caption{Feynmann diagrams for the leading Higgs boson production modes at the LHC \cite{Djouadi}.}
    \label{fig:FeynmannHiggs}
\end{figure}

The LHC Cross-section Working Group produces Higgs measurement predictions on cross-sections, branching ratios, and pseudo-observables. Four CERN reports bring together Higgs recommendations for current and planned LHC efforts \cite{LHCCrossSectionWG}. Figure \ref{fig:HiggsCrosssection} shows the Higgs cross-section for the main production modes as s function of the Higgs mass and the collision center-of-mass energy. The Standard Model doesn't predict the Higgs mass, but with our experimental knowledge of that $m_H$ is about 125GeV, the calculations can be focussed on the region of interest. Examining cross-section as a function of center-of-mass demonstrates the increase in statistics for events of interest that running the detector at higher energy levels can accomplish. It's also clear that gluon-gluon fusion Higgs production is the leading production mechanism by far. The cross-section is currently known at NNLO in QCD with NLO EW corrections. The second highest production cross-section is from vector-boson-fusion (VBF). As seen in the Feynman diagram, two outgoing quarks are produced in the interaction. These quarks produce two hard jets in the forward region, with the Higgs boson appearing between them. To leading order VBF Higgs production is solely electroweak and QCD correction (calculated at NLO) have a smaller impact than in ggF. NLO EW corrections are also applied. As a result, VBF theoretical uncertainties are much smaller than those on ggF. Vector boson associated Higgs production through a $W/Z$ boson are less common the VBF but also dominated by electroweak processes with a small QCD correction (NNLO). Finally, associated production with top and bottom quarks is shown, though these are quite rare and have high NLO and NNLO QCD corrections. 

\begin{figure}[H]
    \centering
    \subfloat[Higgs production cross-section by $m_H$]{{\includegraphics[width=0.45\textwidth]{Pictures/HiggsCrosssectionMH.png} }} %
    \qquad
    \subfloat[Higgs production cross-section by$\sqrt{s}$]{{\includegraphics[width=0.45\textwidth]{Pictures/HiggsCrosssectionCOM.png} }}%
    \caption{Higgs production cross-sections over Higgs mass at center-of-mass energy 14TeV (left) and over center-of-mass energy for a Higgs mass of 125GeV \cite{LHCCrossSectionWG}}%
    \label{fig:HiggsCrosssection}%
\end{figure}

Theoretical uncertainties as shown as colored bars in \ref{fig:HiggsCrosssection} are created from choice of PDFs, and renormalization and factorization scales. Parton distribution functions (PDFs) are described in more detail by the PDF4LHC working group \cite{PDF4LHC}. This group performs studies of PDFs and their predictions at the LHC and makes recommendations for methods of estimating PDF uncertainties. 

\begin{figure}[H]
        \centering
    \includegraphics[width=0.5\textwidth] {Pictures/HiggsBranching.png}\hspace{1cm}
    \caption{Predicted branching ratios for the Higgs boson at the LHC as a function of Higgs mass \cite{LHCCrossSectionWG}.}
    \label{fig:HiggsBR}
\end{figure}

Since the Higgs boson coulplings are directly proprtional to the masses of said particles, the Higgs decays most readily into the heaviest particles possible. Figure \ref{fig:HiggsBR} shows key decay mode branching ratios near the experimentally known Higgs mass. While the branching ratios demonstrate the relative abundance of each Higgs decay, these don't translate directly into their ease of discovery or measurement. The current status of Higgs boson coupling and cross-section measurements for each decay mode will be detailed in the next section. The Higgs boson discovery was made through a combination of searches in many channels though dominated by $H\rightarrow ZZ^*$, $H\rightarrow\gamma\gamma$, and $H\rightarrow WW$ \cite{Higgsdiscovery}. This is because though other decay branching ratios are higher, like $H\rightarrow b\bar{b}$, the background associated with this decay is much higher. As previously mentioned, the proton-proton collisions create large amounts of QCD jets that are difficult to discern from target QCD processes. Because of this, Higgs decays to quarks and gluons are particularly difficult and those with leptonic products (ZZ decays to 4 leptons, decays directly to photons, and $WW$ decays to two leptons and two neutrinos) are much easier to reconstruct. As energy and integrated luminosity increased during LHC runs 1 and 2, measurements of even rare and background-heavy Higgs decay channels were able to be made.  This thesis focuses on the decay of $H\rightarrow W^+W^-\rightarrow \ell\nu\ell\nu$ through VBF production and in the last section of this chapter I will motivate the choice of these parameters for probing new physics beyond the SM.   

\section{Brief history of SM tests}
The Standard Model can be said to be the most rigorously tested physical theory ever developed. The history of the SM could start with any number of physicists well before the formalism of the theory itself. As far back as the fifth century BC there have been proponents of the idea that matter is composed at its most fundamential nature, discrete "particles", but this idea was only tested beginning in the nineteenth century when chemists were able to physical evidence of atoms and their structure. The first of the gauge theories, QED, was invented in the 1930s but only calculated to first order. Renormalization theory, invented simultaneously by Feynman, Schwinger, and Tomonaga in the 1940s, 
es made calculations of higher order QED results possible. Following this advancement, QED was verified experimentally with incredibly high precision. 

Physicists next attempted to understand and formalize the other fundamental forces- strong and weak- in the same way. Symmetries for these theories were not as easy to find as that of QED and in 1954 the first new gauge theory for QCD was proposed by Shaw, Yang, and Mills. Though ultimately incorrect it led to proposals for a gauge theory of weak interactions by Schwinger in 1956 (unifying weak and electromagnetic interactions with photons and massive $W^\pm$ bosons). Glashow added a fourth boson, $Z$, to the theory, but the problem of a broken symmetry necesary to give mass to the $W/Z$ bosons remained. Spontaneous symmetry breaking was a known and tested concept, but its use in the theory led to unwanted massless "Nambu-Goldstone" bosons, a problem some thought was inevitable in using SSB in the gauge theory. In 1964 Englert and Brout then Higgs and a few months later Guralnik, Hagen, and Tribble all published papers with the same conclusion- the Goldstone theorem can be applied to gauge theories and massive bosons can "eat" the Nambu-Goldstone bosons to gain mass. This then creates a new scalar field whose particles are termed Higgs bosons. A few years later Weinber unified these ideas into the electroweak theory we know now. The theory was confirmed many times over at experiments over the next decades. 

QCD

Cosmic rays, accelerators, discoveries 

Higgs discovery

HWW cross section- ATLAS/CMS

HWW differential
\section{Measurement motivation?}
Why HWW? 
Hints from phenomology, why differential? 
Discuss the future aka the fraction of vbf to SM

