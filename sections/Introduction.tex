The Standard Model describes particles and their interactions at fundamental scales and provides the theoretical tools to test the limits of that knowledge. No significant deviation has thus far been observed between Standard Model predictions and experimental results, but the search continues. Our observable universe shows evidence of physics beyond the current theory, and the search for evidence of what this new physics might be motivates continued measurements of Standard Model parameters. Chapter 2 describes the theoretical motivations for this thesis and the key experimental results that preceed it. 
 
The ATLAS detector precisely measures a wide-range of Standard Model parameters. Chapter 3 summarizes the design and performance of the ATLAS detector. This thesis uses prescriptions for reconstruction, identification, isolation and measurements of efficiency and scale from a number of dedicated ATLAS performance groups. Chapter 4 discusses methodology for these recommendations and their particular use in the final state of our measured decay mode. 

 This thesis focuses on the fiducial cross-section of vector boson fusion Higgs bosons in the $WW^*$ decay channel. This decay channel has a final state that requires accurate reconstruction of leptons and jets within the detector and greatly benefits from increased statistics available from the latest data. Chapter 5 summarizes the dataset, simulations and observables used in the cross-section measurement as well as the signal event selection. Chapter 6 describes each major background and uncertainty that affects the final measurement, and Chapter 7 details the statistical analysis and final results.  

The $H\rightarrow WW^*\rightarrow \ell\nu\ell\nu$ cross-section measurement is the cornerstone of this thesis and has been my primary focus for the past two years. I also had the opportunity to work on detector upgrades and muon performance studies during my Ph.D. studies so these topics are described with particular attention in Chapters 3 and 4. 

All measurements and studies included here were made possible through the advice and software programs from countless others. All figures contained in this thesis are cited from either publications or individuals unless I was the direct author.

