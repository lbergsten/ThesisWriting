 
The Large Hadron Collider (LHC) is a proton-proton storage ring operating at CERN and for its 9 years of operation, it has been the world's highest energy particle collider. 
%The LHC first began operation in 2008 but following a magnetic quench incident it had to be repaired and adjusted. The first data-taking commenced in 2009 \cite{Rossi_2010}. 
During LHC operation thus far, protons have collided with increased center-of-mass approaching the design energy of 14 TeV. Instaneous luminosity has also successively increased, surpassing design instaneous luminosity of $1\times10^{34}$cm$^{-2}$s$^{-1}$ in 2018 to reach $2\times10^{34}$cm$^{-2}$s$^{-1}$\cite{CERNnews1}. The overall data recorded in the ATLAS detector totals more than $10^{16}$ collisions. Operation of the LHC has led to the discovery of the Higgs boson and some of the most precise measurements of its properties including the coupling of the Higgs boson to bottom quarks \cite{Aaboud_2018_0}, $W$ and $Z$ bosons \cite{Aaboud_2019}, \cite{Aaboud_2018}, photons\cite{Aaboud_2018_2} and tau leptons \cite{Aaboud_2019_2}. The LHC has also facilitated searches for new physics over a wide parameter space, setting confidence level exclusion limits on masses of supersymmetric particles like squarks, gluinos and neutralinos \cite{ATLAS-CONF-2019-040}. 

The LHC can run continuosly for a few years before detector components need to be repaired and replaced. The schedule of data-taking consists of long periods of data accumulation (Run 1 from 2015-1016, and Run 2 from 2017-2018) paired with long shutdown periods. The LHC is set to begin Run 3, in which design center-of-mass energy should be reached, in 2021. Following Run 3 detector upgrades will be installed during a long shutdown. Then the High-Luminosity LHC (HL-LHC) will begin colliding protons with unprecedented ($10\times$) luminosity in 2027 \cite{CERNnews2}. The HL-LHC and its goals will be explained further later in this chapter. Suffice to say LHC-physics is progressing quickly and promises exciting developments in the near future. 

In a brief explanation of the LHC operation, one could begin with the small volume of $\approx 10^{11}$ protons that are accelerated per bunch. Linac-2 is the primary accelerator for CERN colliders and has been since the early 1990s \cite{LHCInjector}. This injects protons at 50 MeV into the Proton Synchrotron Booster (PSB) where they are further accelerated to 1.4 GeV. In the the Proton Synchrotron (PS), the protons are separated into bunches with a spacing of 25 ns and are futher accelerated to 25 GeV before being extracted to the Super Proton Synchrotron (SPS), where they reach 450 GeV. Finally the bunches of protons enters the LHC, where they are accelerated to their final energy of 6.5 TeV. Linac 2, PSB, PS, and SPS were all operational accelerators before the LHC era though each had to be majorly upgraded to handle the energy and beam intensity required for LHC collisions \cite{LHCInjector}. 

The LHC layout mimics that of the Large Electron Positron collider (LEP) that was housed in the same tunnels. Figure \ref{fig:LHClayout} shows the positioning of each experiment at the LHC as well as injection systems and other features. Once proton bunches enter the LHC in two opposing beams, they are accelerated with radio frequency (RF) systems. Located at Point 4 in the LHC schematic, the system consists of 16 RF cavities operating at twice the frequency of the SPS injector. RF cavities are metallic chambers containing oscillating electromagnetic fields; in the LHC this oscillation frequency is 400 Mz. The tuning of this frequency ensures that protons of the ideal energy are not accelerated further and simply maintain their momentum while particles arriving in an RF cavity slightly before or after will be decelerated or accelerated toward the ideal proton energy. This acceleration process can also be used to split beams of protons into discrete bunches, and this is first done with RF cavities in the PS. After proton bunches have circled the LHC approximately 1 million times (15 minutes), peak energy is reached and collisions can commence \cite{radiofrequency}.

\begin{figure}[!h]
        \centering
    \includegraphics[width=.6\textwidth]{Pictures/LHClayout.PNG}
    \caption{LHC layout \cite{LHCref}}
    \label{fig:LHClayout}
\end{figure}

Superconducting magnets in the LHC main dipoles create a magnetic field of $\approx 8$ T to bend the proton beams into the circlular path of the collider. Figure \ref{fig:dipolemagnet} shows the flux in a dipole cross-section. The opposing direction beamlines are shown centered and the flux is shown to be high (and directionally opposed) in the center of each beam. To maintain these fields, the magnets operate at below 1.9K. Pressurized superfluid helium chosen for its low visosity and high specific heat cools the dipole magnets. Once the two LHC rings are filled from the SPS the center-of-mass energy of the beams increases until it reaches peak energy after about 28 minutes. Finally, proton bunches separated by 25ns collide simultaneously in each detector.  

\begin{figure}[!h]
        \centering
    \includegraphics[width=.6\textwidth]{Pictures/dipolemagnet.jpg}
    \caption{ Flux within an LHC dipole cross-section \cite{LHCref}}
    \label{fig:dipolemagnet}
\end{figure}

\section{ATLAS, A Toroidal LHC ApparatuS}
\hspace{20pt} The LHC creates proton-proton collisions at the rate and energy necessary for pushing the boundaries of particle physics, but identifying and reconstructing the tracks of such energetic particles is no mean feat. A Toroidal LHC ApparatuS (ATLAS) and the Compact Muon Solenoid (CMS) are multi-purpose detectors built to search for a wide range of particle interactions and the measurement of their properties. Both experiments measured a particle consistent with the Higgs boson in 2012 and their agreement was a key verification of the discovery. The following sections describe each major component of the ATLAS detector so to highlight their role in the measurement of Higgs$\rightarrow WW \rightarrow \ell\nu\ell\nu$. 

ATLAS utilizes a coordinate system with its origin at the center of the detector (the ``interaction point") and has a z-axis along the beam pipe. The x-axis points from the interaction point to the center of the LHC ring, and the y-axis points upward. The experiment uses cylindrical coordinates $(r, \phi)$ where $\phi$ is the azimuthal angle around the beam pipe. The pseudorapidity and the transverse momentum are defined in terms of the polar angle $\theta$ as $\eta = -\ln( \tan(\theta/2))$ and $p_T = p\sin\theta$. 
\begin{figure}[!h]
	\centering     \includegraphics[width=.7\textwidth]{Pictures/ATLASdetector.PNG}
    \caption{Computer-simulated ATLAS detector schematic \cite{detector}}
\end{figure}

The Inner Detector (ID) detects charged particles with $|\eta| < 2.5$ operating in a 2 T solenoidal field. It consists of $3$ layers of pixel sensors, $4$ layers of silicon strips, and $72$ straw layers of transition radiation tracker modules. The ID describes particles closest to the interaction point and locates track parameters with great resolution due to its high granularity \cite{detector}. 

The ATLAS detector contains 3 superconducting magnet systems- the central solenoid, barrel toroid, and 2 end-cap toriods. The central solenoid provides a magnetic field for the inner detector while the toroids create a strong magnetic field for for the muon detector. These magnets were built to create the largest possible uniform magnetic field to maximize the momentum resolution on particle tracks. They also need to use as little material as possible so as to not unduly influence particles in the detector. The toroids in the barrel and endcap each have 8 coils and create a 4 T magnetic field while the central solenoid creates a 2 T magnetic field in the inner detector. Combined the magnet systems contain $>$100 km of superconducting wire which are cooled to working temperatures below 5K \cite{detector}. 

The Muon Spectrometer precision chambers provide muon momentum measurements at a high resolution over a wide range of $p_T$. The MS consists of $3$ layers of Monitored Drift Tube chambers covering $|\eta| < 2.7$ and an inner layer of Cathode Strip Chambers with $|\eta| > 2.0$. In addition, it includes trigger chambers that contain $3$ layers of Resistive Plate Chambers ($|\eta| < 1.05$) and $3$ layers of Thin Gap Chambers ($1.05 < |\eta| < 2.4$). As the outermost subdetector, the MS provides precise muon momentum measurements along the muon trajectory and the muon chambers are located with a precision of under $60$ $\mu m$. The MS also contains a system of three superconducting toroidal magnets each with eight coils providing a magnetic field with a bending integral of up to $6$ Tm \cite{detector}. 

Calorimeters provide detailed information about the energy deposited as particles pass through. Electromagnetic calorimeters detect and halt the motion of electrons and photons while the hadronic calorimeter does the same for hadrons. Muons and neutrinos are able to pass through the calorimeters to the MS. The electromagnetic and hadronic calorimeters, made of liquid Argon and scintillating tiles respectively, are able to pass information from the location of energy deposits to the various idenfication and reconstruction algorithms \cite{detector}. 

\section{The High-Luminosity LHC and Inner Tracker (ITk)}
The LHC succeeded in its paramount goal of discovering the Higgs boson in 2012. Its continuous operation at higher energy and luminosity has led to more rigorous measurements of Higgs boson properties as well as searches for new physics beyond the Standard Model. While more data collection is planned in Run-3 starting in 2021, new colliders and detectors take decades to design, develop and build, so the plans for upgrading the current detectors are well underway. The High Luminosity LHC will operate at 14 TeV starting in 2027. The HL-LHC will begin with $5-7\times$ the luminosity of the LHC and peak at the design instaneous luminosity of $10\times$ the LHC, or $12.6\times10^{-34}$cm$^{-2}$s$^{-1}$. This huge increase in number of collisions requires substantial upgrades to the LHC including new 11-12 T superconducting magnet systems, compact superconducting cavities for beam rotation and phase control, and new technology beam collimation \cite{HLLHCYellow}. 

Just as the LHC had to be re-designed, so too do all the experiments to be able to handle the much higher luminosity. The detectors must be built to withstand more radiation, as the increased collision rate also means a high radiation rate, especially closest to the beamline. They also have to provide greater granularity to be able to reconstruct tracks with good enough resolution that individual tracks can be discriminated. Finally, they have to faster to be able to deal with increased pile-up. Pile-up is caused by high numbers of collisions occuring at each bunch crossing. When there is a large amount of pile-up it becomes difficult to trace which particle tracks come from the same interaction point. Finally, the increased data in and of itself creates a complex problem for the detectors to solve. The trigger system must quickly select and store the events that may hold interesting information. 

Detectors for high energy colliders are not built often - expensive and time-consuming to design and test, they are made to last at least a decade. I was lucky to have the opportunity to work on ATLAS detector research and development during the 1.5 years I worked at Brookhaven National Laboratory during my Ph.D. Though my thesis is not directly related to this work, it was formative and extremely interesting, so I touch on this in the next section. Because I worked on the new ATLAS Inner Detector for the HL-LHC (termed Inner Tracker or ITk) I will discuss solely this sub-detector and the particular role I played in its assembly.

\subsection{Inner Tracker (ITk)}
The Inner Tracker is planned to be an all-silicon detector that will completely replace the current Inner Detector.  While the current ID has been extremely successful during Runs 1 and 2 (and will certainly continue to be in Run 3), it does not have the capacity to withstand the radiation and pile-up conditions of the HL-LHC. The ITk is designed to operate for 10 years under an instantaneous luminosity of 7.5$\times10^{-34}$cm$^{-2}$s$^{-1}$ with 25 ns between bunch crossing. This will result in 1,000 fb$^{-1}$ and average pile-up up to $<\mu>=200$ \cite{ITktech}. The current solenoid magnet will remain in the detector with a 2 T magnetic field. The ITk will consist of an innermost section with silicon pixels and an outermost section of silicon strips. The pixel detector will contain four barrel layers and six forward region disks, while the strip detector will contain five barrel layers and seven disks. The rapidity range matches the coverage of the Muon Spectrometer with $|\eta|<2.7$. This layout is shown in \ref{fig:ITklayout}. 
\begin{figure}[!h]
        \centering
    \includegraphics[width=.6\textwidth]{Pictures/ITklayout.png}
    \caption{ ITK layout as defined in ITk Technical Design Report \cite{ITktech}}
    \label{fig:ITklayout}
\end{figure}

\subsubsection{Building ITk Strip barrel staves}
At Brookhaven National Laboratory, I made key contributions to the ITk Strip barrel stave assembly effort. The goal of stave assembly is to glue silicon modules to carbon fiber stave cores within a 25 $\mu$m tolerance. Brookhaven is responsible for assembly of 200 ITk staves and their accurate assembly is necessary for the ITk to reduce uncertainty on track positions as well as to ensure a symmetric detector. I was tasked with co-creating a stave assembly software system through LABView to automatically calibrate required module positions, apply a layer of adhesive gel, and guide a user in accurately placing a module into its specified location. This project was highly collaborative and evolved further after I left the laboratory, but the overall process remains unchanged.

The basic design of the Inner Tracker for both barrel and endcap components is the same - a carbon fiber core (containing titanium cooling pipes) is covered on each side with co-cured kapton service tapes. The carbon fiber core is designed to reduce material inside the detector and the similar design in the barrel and endcap adds to simplicity. Silicon modules are glued to stave cores. Similar silicon strip detectors have been used previously in both ATLAS and CMS, but have never covered so much fiducial area. The modules consist of one silicon sensor and one or two low-mass PCB's (hybrid) which host ASICs. Module design has optimized producibility and low cost while maintaining readout goals. Overall module design is the same in barrel and endcap regions, while strip lengths and geometries vary. Components of a short-strip barrel module are shown in \ref{fig:module}.
\begin{figure}[!h]
        \centering
    \includegraphics[width=.4\textwidth]{Pictures/ITkmodule.png}
    \caption{Short-strip barrel module components \cite{ITktech}}
    \label{fig:module}
\end{figure}
 
Each barrel stave core needs to be ``loaded" with 14 modules, as shown in the assembled electrical prototype in \ref{fig:stave}. 

\begin{figure}[!h]
        \centering
    \includegraphics[width=.8\textwidth]{Pictures/electricalstave.png}
    \caption{Electrical stave prototype at Brookhaven National Laboratory (G. van Nieuwenhuizen)}
    \label{fig:stave}
\end{figure}

Brookhaven National Laboratory is one of two sites responsible for assembling barrel staves. Assembly procedures have been tested with the production of numerous prototypes including a thermomechanical double-sided stave and a fully operational electrical stave. The thermomechanical prototype was later used for various thermal tests, including IR imaging. The electrical stave was used for testing the full electrical read-out. Stave assembly is composed of three main parts: system calibration, module placement, and survey of results. 

Staves at BNL are assembled on a granite table housing an Aerotech XYZ Stage accurate to the micron level. The stage is equiped with a 10-megapixel camera that gives real-time feedback to a nearby computer and a glue dispenser. The stave assembly software system is implemented by a user who interacts with a LabView GUI and monitors progress. The stave is fixed to optical rails drilled into the granite table. In order to accurately place modules in their correct positions a series of calibrations need to be completed including camera calibrations to test the optimal working point, focus, and pixel-to-micron conversion. Next, the position of the stave with relation to the XYZ stage needs to be calibrated. Tranforming coordinates of the XYZ stage to that of the stave requires locating a fixed point on the stave core as well as the angle of the stave relative to the XYZ stage. Pattern matching algorithms find the exact locations of particular features on the stave core and allow calculation of required positions for all modules based on specifications. Once specified, module positions are calculated, calibrations are completed and it is time to apply glue and afix modules. 

Next, an epoxy (SE4445) is loaded into the glue dispenser on the XYZ-stage which is connected to a vaccuum controlled by the LABView software system. The epoxy is automatically dispensed in lines to cover $\approx$ 60\% of area under the module. Then the module is lifted with a custom-made ``pick-up" tool which uses vacuum applied to module corners to hold the module in place and move it to the needed position along the optical rails. Using real-time feedback from the software system and its pattern matching algorithm, the user is directed on how to finetune module position using knobs on the ``pick-up" tool. Markings etched in the silicon sensor at each corner are used to position the module accurately. The output of the module alignment GUI is shown in \ref{fig:modulealignment}. When the module is within specifications it is lowered into position above the epoxy and held in place for 24 hours until the glue has completely dried.

\begin{figure}[!h]
        \centering
    \includegraphics[width=.6\textwidth]{Pictures/labviewscreen.png}
    \caption{GUI interface showing etched marking on module corner located in real-time to guide user on how to adjust module position (H. Herde)}
    \label{fig:modulealignment}
\end{figure}

After the glue has set, a final survey of module positions is recorded using pattern matching to find positions of etched markings on each module corner. These results are saved into an ITk database and checked for any biases. After module placement on the stave is complete, the loaded stave is moved to another station in the lab for wirebonding so that all data from the modules can be read-out to stave-wide electronics. The module assembly system has been successful at placing modules accurately for all prototypes, achieving specification requirements for almost all modules. Results of the first prototype stave's module placement are shown in \ref{fig:placementresults}. While a few module corners are slightly out of the ideal range, the majority are well within specifications. Throughout prototype assembly issues and inefficiencies were found and corrected. New hardware, like an improved glue system and temperature monitoring, were also added. The methods described continue to be in use now and will be utilized for the production of 200 ITk staves at BNL starting in 2021. 

\begin{figure}[!h]
        \centering
    \includegraphics[width=.8\textwidth]{Pictures/placementresults.png}
    \caption{Histograms show difference between ideal and final position of each module corner. Left shows difference from specification in X and right in Y (P. Bhattarai)}
    \label{fig:placementresults}
\end{figure}

\subsubsection{IR Testing of ITk Strip barrel staves}
The first full US stave prototype was the thermo-mechanical stave built in the summer of 2017.  Building this stave was the first test of stave assembly procedures and the results proved that the module placement algorithm we developed worked. This stave was also used to test the thermal and mechanical properties of a fully loaded barrel stave. Multiple studies were conducted, including thermal measurements using thermistors and IR imaging, thermal cycling and thermal shock tests, mechanical studies, and bending tests. I will give a short summary of IR imaging tests, as these were another focus of my time at BNL. 

The thermo-mechanical stave consists of 13 modules mounted on each side. The modules used are thermo-mechanical, which means that instead of the usual readout chips their hybrids employed copper resistors to mimic the power dissipation and location of the chips. The powerboard can vary the TM hybrid power dissipation of each module individually. Three thermistors were mounted on each TM module- one on the DC-DC converter and one on each of the two hybrids. A custom End-of-Substructure (EoS) card is attached to a RaspberryPi and Arduino to power on or off each module.  
 
Thermal testing of barrel staves had a few main goals: to validate Finite Element Analysis (FEA) simulations by testing that all temperature trends are as we expect, to make sure that individual modules do not exhibit abnormal thermal behavior, and finally to check that loaded staves can cope with large changes in temperature they might face during operation. I will highlight a few key results which demonstrate that each of these goals have been accomplished. 

Thermal measurements were taken both through the mounted thermistors on each module and through IR imaging. IR imaging provides information about the entirety of the loaded stave, rather than at just a few module positions so provides a more complete picture. The loaded stave was spray-painted black with a high emissivity, low conductivity black paint since silicon is transparent to the IR camera spectrum (8-14$\mu$m). In order to image the entire stave core, the IR camera was attached to rails above and pulled at a constant speed with an external motor as it recorded a video. The frames were then stitched together into one image. A section of the painted stave is shown in \ref{fig:paintedstave}.  

\begin{figure}[!h]
        \centering
    \includegraphics[width=.4\textwidth]{Pictures/paintedstave.png}
    \caption{Portion of the thermo-mechanical stave after spray-painted for increased emissivity}
    \label{fig:paintedstave}
\end{figure}

FEA simulations for the thermal performance of a stave were completed by Prof. Graham Beck at Queen Mary University of London. These calculations quickly became intractable if convection was included so conditions of the stave and coolant were adjusted to minimize convective constributions or make sure that the total electrical power and power absorbed into the coolant were identical. At BNL we adjusted the coolant temperature until we obtained the convective power minimization and then recorded module temperatures under these conditions. These results were compared to the FEA simulations by averaging hybrid temperatures recorded through IR imaging and recording NTC thermistor readings. These comparisons are shown in \ref{fig:FEAcompare}. The measurements show very good agreement with FEA calculations, within 5\% of the expected values.

\begin{figure}[!h]
        \centering
    \includegraphics[width=.4\textwidth]{Pictures/FEAcompare.png}
    \caption{IR measurements, NTC thermistor measurements, and FEA simulations for the TM stave are compared. Agreement with FEA simulation within 5\%.}
    \label{fig:FEAcompare}
\end{figure} 

During module assembly some slight variations were tested, including varying glue thickness below modules, glue curing time, and FEAST versions. Modules with and without these variations were compared at varying coolant temperatures and output power settings. Overall, no significant difference in module temperature change was observed for any of these assembly modifications. Stave thermal properties are thus robust to such assembly modifications. Figure \ref{fig:IRimagetotal} shows a full IR image of the fully loaded stave. It is clear that there are no obvious module-to-module variations in silicon, hybrid, or FEAST temperature. The module sensors increase in temperature as they get closer to the EoS, which is expected since it dissipates power to the stave.

\begin{figure}[!h]
        \centering
    \includegraphics[width=.8\textwidth]{Pictures/IRimagetotal.png}
    \caption{IR image of fully loaded thermo-mechanical stave}
    \label{fig:IRimagetotal}
\end{figure} 
 
The thermo-mechanical stave was pushed to limits beyond what we would expect loaded staves to encounter during operation and never exhibited unexpected behavior. Thermal cycles, thermal shocks and bend tests showed the loaded stave to be robust against temperature variation and that the carbon fiber core is as stiff as it was prior to loading. Another test was how neighboring modules would perform if one module malfunctioned and was powered off. Figure \ref{fig:moduleoff}(top) shows the temperature of each module when one of them (fourth to the left) is turned off. The rest of the modules continue to operate normally and temperature changes from the unpowered module do not propagate very far. The bottom image in the figure shows the reverse of the stave when a module is powered off (fourth from the right). The temperature effects are greater on the module directly below than those adjacent to, which is expected due to material differences. 

\begin{figure}[!h]
        \centering
    \includegraphics[width=.7\textwidth]{Pictures/moduleoff.png}
    \caption{IR image of fully loaded thermo-mechanical stave}
    \label{fig:moduleoff}
\end{figure}

My experiences on ATLAS detector upgrades for the HL-LHC provided context for the bulk of my thesis research. This project provided me with in depth and hands-on knowledge of the ATLAS detector and its component parts, as well as the scale of effort required to build new detector components. I have an abiding appreciation for the people and technology necessary for data-taking at the LHC, both of which make measurements like the Higgs cross-section detailed in this thesis possible.
